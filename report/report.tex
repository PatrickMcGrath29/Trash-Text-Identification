%!TEX TS-program = xelatex
%!TEX encoding = UTF-8 Unicode

\documentclass[12pt]{article}

\usepackage{amstext}
\usepackage{array}
\usepackage{enumerate}
\usepackage{fontspec}
\usepackage{multicol}
\usepackage[square,sort,comma,numbers]{natbib}

\renewcommand{\baselinestretch}{1.2}
\renewcommand{\thesection}{\arabic{section}.}
\renewcommand{\thesubsection}{\thesection\arabic{subsection}}

\newcolumntype{L}{>{$}l<{$}}
\setmainfont{Linux Libertine O}

\title{Using Textual Features to Identify Document Reliability}
\author{Julien Cherry \and Patrick McGrath}
\date{April 26\textsuperscript{th}, 2019}

\begin{document}
	\maketitle
	\section{Problem Description}

	\section{Problem Formulation}

	\begin{itemize}
		\item Given the author, title, and text of an article, determine if the article is likely to be reliable or not
	\end{itemize}

	\begin{tabular}{l | L}
		Dataset        & D = \{\langle \text{title}_1, \text{author}_1, \text{text}_1, \text{reliability}_1 \rangle \ldots\} \\
		Training set   & T_a = randomSample_{0.5}(D) \\
		Testing set    & T_b = D \setminus T_a \\
		Classifier     & C = trainClassifier(T_a) \\
		Sample article & a_s \not\in D = \langle \text{title}_s, \text{author}_s, \text{text}_s \rangle \\
		Prediction     & p_s \in \{\text{reliable}, \text{unreliable}\} = predict(C, a_s) \\
	\end{tabular}

	\section{Methodology}

	\begin{itemize}
		\item 18,285-article Kaggle dataset
		\begin{itemize}
			\item Use 50\% for training, 50\% for testing
		\end{itemize}
		\item Clean data
		\item Vectorize data: generate (token, count) tuples
		\item Weight features with TF–IDF
		\begin{itemize}
			\item Prioritize words that are more unique to a given document
		\end{itemize}
		\item Train model (pick one)
		\begin{itemize}
			\item Naïve Bayes
			\item Stochastic Gradient Descent
			\item Passive Aggressive
		\end{itemize}
		\item Use model to predict reliability of test dataset or sample article
	\end{itemize}

	\subsection{Passive Aggressive Classification}

	\begin{itemize}
		\item Online algorithm, streaming data
		\begin{itemize}
			\item Consume example, update classifier, discard example
		\end{itemize}
	\end{itemize}

	\section{Empirical Results}
	\section{Analysis}

	\begin{itemize}
		\item Objective: Identify misinformation
		\item Classification based on words in article
		\begin{itemize}
			\item Doesn’t account for context of words
		\end{itemize}
		\item Clear bias in dataset
		\begin{itemize}
			\item Overfitting?
		\end{itemize}
		\item Correlation !== Causation
	\end{itemize}

	\section{Discussion}
	\section{Future Work}

	Work in this area could be used to inform social media users whether linked articles are likely to be reliable or not. However, its implementation need be informed---Pennycook et al. found that labelling disputed articles with a warning causes unlabelled articles to appear more accurate. Crucially, they found that this \textit{Implied Truth Effect} is reversed when verified articles are also labelled \cite{pennycook}.

	\begin{itemize}
		\item Investigate what kinds of articles are misclassified
		\item Tweak specificity and precision of classifier
		\item Use n-grams, Word2Vec, Doc2Vec
		\begin{itemize}
			\item Improve context around words, instead of standalone words
		\end{itemize}
	\end{itemize}

	\section{Reflection}

	\begin{itemize}
		\item Have a more established definition of what misinformation is
		\item Opportunity to debug classification process
		\begin{itemize}
			\item Identify improvements
			\item Adjust features
		\end{itemize}
	\end{itemize}

	\bibliographystyle{plainnat}
	\bibliography{references}
\end{document}
