%!TEX TS-program = xelatex
%!TEX encoding = UTF-8 Unicode

\documentclass[12pt]{article}

\usepackage{fontspec}
\usepackage[square,sort,comma,numbers]{natbib}

\renewcommand{\baselinestretch}{1.2}
\renewcommand{\thesection}{\arabic{section}.}

\setmainfont{Linux Libertine O}

\title{Using Textual Features to Identify Document Reliability}
\author{Julien Cherry \and Patrick McGrath}
\date{April 26\textsuperscript{th}, 2019}

\begin{document}
	\maketitle
	\section{Problem Description}
	\section{Technical Problem Statement}
	\section{Methods}
	\section{Empirical Results}
	\section{Analysis}
	\section{Discussion}


	\section{Future Work}

	Work in this area could be used to inform social media users whether linked articles are likely to be reliable or not. However, its implementation need be informed---Pennycook et al. found that labelling disputed articles with a warning causes unlabelled articles to appear more accurate. Crucially, they found that this \textit{Implied Truth Effect} is reversed when verified articles are also labelled \cite{pennycook}.

	\bibliographystyle{plainnat}
	\bibliography{references}
\end{document}
